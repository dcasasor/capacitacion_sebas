\documentclass[11pt]{article}

\usepackage[utf8]{inputenc}
\usepackage[spanish]{babel}
\usepackage{physics}
\usepackage{graphicx}
\usepackage{siunitx}
\usepackage{mathpazo}


\title{Solución de la ecuación de conducción de calor en 2D}
\author{Daniel Casas-Orozco}
\date{\today}

\begin{document}

\maketitle

\section{Suposiciones}

\begin{itemize}
	\item El sistema se puede describir en coordenadas cartesianas (coordenadas rectas)
	\item Hay generación uniforme de calor en el sólido
	\item El proceso ocurre en estado transitorio
	\item Sólo hay cambios significativos de temperatura en los ejes $x$ y $y$ (conducción 2D)
	\item Las propiedades del material son constantes en el tiempo y con la posición
\end{itemize}

\section{Desarrollo del modelo}

Con las suposiciones mostradas en la sección anterior, la ecuación de conducción de calor se puede escribir como:

\begin{equation} \label{eq:conduction_eq}
	\pdv[2]{T}{x} + \pdv[2]{T}{y} + \frac{\dot{q}_{gen}}{k} = \frac{1}{\alpha} \pdv{T}{t}
\end{equation}

La idea ahora es expresar la Ec. \eqref{eq:conduction_eq} en diferencias finitas. Las segundas derivadas en el lado izquierdo de la Ec. \eqref{eq:conduction_eq} se pueden escribir como:

\begin{equation} \label{eq:second_deriv}
	\begin{aligned}
		\pdv[2]{T}{x} & \approx \frac{T_{m + 1, n} - 2 T_{m, n} + T_{m - 1, n}}{\left(\Delta x \right)^2} \\
		\pdv[2]{T}{y} & \approx \frac{T_{m, n + 1} - 2 T_{m, n} + T_{m, n - 1}}{\left(\Delta y \right)^2}
	\end{aligned}
\end{equation}

La derivada temporal se puede expresar como:

\begin{equation}
	\pdv{T}{t} \approx \frac{T_{m, n}^{i + 1} - T_{m, n}^i}{\Delta t}
\end{equation}

Si se asume una formulación \emph{explícita} en el tiempo, las derivadas mostradas en las Ecs. \eqref{eq:second_deriv} se escriben con el superíndice $i$, con lo cual la Ec. \eqref{eq:conduction_eq} queda formulada en diferencias finitas como:

\begin{equation}
	\frac{T_{m + 1, n}^i - 2 T_{m, n}^i + T_{m - 1, n}^i}{\left(\Delta x \right)^2} + \frac{T_{m, n + 1}^i - 2 T_{m, n}^i + T_{m, n - 1}^i}{\left(\Delta y \right)^2} + \frac{\dot{q_{gen}}}{k} = \frac{1}{\alpha} \frac{T_{m, n}^{i + 1} - T_{m, n}^i}{\Delta t}
\end{equation}

\section{Nodos de frontera}

Las condiciones de frontera son de convección en los extremos en contacto con aire: 

\begin{multline}
        \frac{h}{k \Delta x} (T_{\infty} - T_{M, n}) + \frac{1}{k} \frac{T_{M - 1, n} - T_{M, n}}{\left( \Delta x \right)^2} + \\
       	\frac{1}{2 k \left( \Delta y \right)^2} \left( T_{M, n + 1} - 2 T_{M, n} + T_{M, n - 1} \right) \\ +
      	\frac{\dot{q_{gen}}}{2} = \frac{1}{\alpha} \frac{T_{m, n}^{i + 1} - T_{m, n}^i}{\Delta t} 
\end{multline}

Los nodos de frontera tienen la mitad del volumen de un nodo central, con lo cual las Ecs. \eqref{boundary_convec_x} se escribe como:

\end{document}
